\documentclass[a4paper,10pt]{article}
\usepackage[latin1]{inputenc}
\usepackage{color}
\usepackage{colortbl}

%opening
\title{Astronomie Praktikum - Theodolit}
\author{Ramon Gfeller, Andreas Zuber}

\begin{document}

\maketitle


\begin{abstract}
Ziel dieses Versuches war, den Stoff aus der Vorlesung ``Astronomie I'' in die Praxis
umzusetzen. Im ersten Teil wurden zwei Sterne ausgew�hlt, welche wir anschliessend mit
dem Thodoliten am Himmel lokalisierten. Im zweiten Teil wurden die Koordinaten zwei
zuf�lliger Sterne gemessen um dann in einem Sternkatalog deren Namen nachzuschlagen.
\end{abstract}


\section{Ausr�stung und Software}
\begin{itemize}
 \item \textbf{Theodolit} (DKMZ-AE 249231) zur Messung von Azimut und Elevation in GON (400ster Bruchteil eines Kreises)
 \item \textbf{Wetterdaten} der Station auf dem ExWi \cite{Wetter}
 \item Programm \textbf{tktrf} der UniBe zur Umrechnung der Koordinaten
    (�quator-Fr�hlingspunkt-System $\leftrightarrow$ Horizont-Ortsmeridian-System).
    Zus�tzlich korrigierte das Programm anhand von Luftdruck, Temperatur und Luftfeuchtigkeit die Refraktion
 \item Programm \textbf{Stellarium}\cite{Stellarium} f�r die Suche zurzeit sichtbarer Sterne
 \item \textbf{Simbad}\cite{Simbad} f�r die Bestimmung eines Sterns anhand seiner Koordinaten
 \item \textbf{Zeitdurchsage} (Nr. 161) der Schweiz
 \item \textbf{Stoppuhr} zur genauen ($\pm0.5s$) Bestimmung des Messzeitpunktes
\end{itemize}


\section{Vorbereitung}
Als Vorbereitung haben wir uns mit den Programmen welche zur Auswertung benutzt werden vertraut gemacht sowie
mehrere Sterne f�r die Beobachtung mittels Stellarium\cite{Stellarium} ausgew�hlt. Dabei war es wichtig das diese
ausgew�hlten Objekte f�r die dauer der Messungen unter einem g�nstigen Winkel sichtbar sind und sich keine hellen Objekte
wie der Mond in unmittelbarer n�he befinden.

\subsection{Sterne f�r die Beobachtung}


\definecolor{tcA}{rgb}{0.627451,0.627451,0.643137}
\begin{center}
\begin{tabular}{llll}
% use packages: color,colortbl
\rowcolor{tcA}
\textbf{Stern} & \textbf{HIP} & \textbf{Rektaszension} & \textbf{Deklination}\\
Capella & 24608 & 5h 16m 41.5s & +45\textdegree  59' 51.0''\\
Hamal ($\alpha$ Ari) & 9884 & 2h 07m 10.5s & +23\textdegree  31' 57''\\
Almaak ($\gamma$ 1 And) & 9640A & 2h 03m 54.0s & +42\textdegree  19' 47.0''\\
Mirphak ($\alpha$ Per) & 15863 & 3h 24m 19.4s & +49\textdegree  51' 40''
\end{tabular}
\end{center}



\section{Durchf�hrung der Messung}
F�r die Zeitmessungen wurde UTC verwendet, die Umrechnung auf UT1 war nicht n�tig, da
der Fehler kleiner war als die Ungenauigkeit des Theodoliten.

Der Theodolit wurde auf das Stativ auf dem Dach des ExWi geschraubt und mit der
integrierten Wasserwaage und den Drehkn�pfen m�glichst horizontal ausgerichtet.

Das Ablesen der Messungen war etwas knifflig. Durch eine Lupe konnten die Skalen
betrachtet werden. Nun musste je nachdem ob Azimut oder Elevation abgelesen wurde der
obere oder untere bewegliche Strich mit einem Drehknopf zwischen die beiden fixen Striche
geschoben werden.

Durch das Okular waren mehrere Fadenkreuze sichtbar. Vermutlich wurde einmal (Teil 2,
Stern 2) nicht das richtige Fadenkreuz verwendet, weshalb der Stern nicht bestimmt werden
konnte.


\subsection{Orientierung am Polarstern}
Als Referenzpunkt f�r die Messungen diente uns der Polarstern. Bei der ersten Messung gieng es also
darum den Polarstern ins Fadenkreuz zu bringen und den Theodoliten entsprechend den Koordinaten einzustellen.

\definecolor{tcA}{rgb}{0.627451,0.627451,0.643137}
\begin{center}
\begin{tabular}{lll}
% use packages: color,colortbl
\rowcolor{tcA}
\textbf{Polarstern (FK907)} & & \\
Zeit & 16:50 & \\
Koordinaten (Grad) & 0 58 38.10 & +47 03 25.55 \\
Koordinaten (GON) & 1.06583 & 47.7142
\end{tabular}
\end{center}


\subsection{Messung 1}
Bei dieser Messung haben wir zwei der von uns ausgew�hlten Sterne Beobachtet. Zuerst
haben wir die Koordinaten des Sterns zur Beobachtungszeit mittels dem Programm ``tktrf''
berechnet.

In einem zweiten Schritt haben wir den Theodoliten auf die Koordinaten eingestellt und den
Durchgang des Sterns durch das Fadenkreuz beobachtet.

beide Messungen waren erfolgreich. Beim zweiten Stern gab es eine kleine Abweichung vermutlich
auf Grund der Eigenbewegung des Sterns welche wir nicht ber�cksichtigt hatten.

TABLE

\subsection{Messung 2}


\begin{thebibliography}{99}

\bibitem{Wetter} \verb+http://www.iapmw.unibe.ch/research/projects/meteo/latest_exwi.html+

\bibitem{Stellarium} \verb+http://www.stellarium.org/+

\bibitem{Simbad} \verb+http://simbad.u-strasbg.fr/simbad/+

\end{thebibliography}

\end{document}
